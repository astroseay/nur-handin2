\subsection{4b}

\lstinputlisting{p4b.py}
\lstinputlisting{p4b.txt}
We use the midpoint rule to calculate the desired integral again, but in terms
of the scale factor $a$. We give the value of the analytic derivative of the
linear growth factor at redshift $z=50$ ($a = \frac{1}{51}$),
\begin{equation}
 \dot D(t) = \frac{dD}{da}\dot a,
\end{equation}
using the integral we calculated numerically. The derivative is,
\begin{equation}
 \dot D(t) = \frac{-15}{4}\Omega_m^2 H_0 \frac{I}{a_f^3},
\end{equation}
where $\Omega_m$ is the matter density, $H_0$ the Hubble constant,
$a_{\textrm{f}}$ the final scale factor, and $I$ the integral given by,
\begin{equation}
 \int_{a=0}^{a=\frac{1}{51}} \frac{\frac{1}{a^3}}
        {{\frac{\Omega_{\textrm{m}}}{a^3}} + \Omega_{\Lambda}},
\end{equation}
where $\Omega_{\Lambda}$ is the cosmological constant.
